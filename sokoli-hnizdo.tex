\setupoutput
   [pdftex]
\enableregime
   [utf]
\definetypeface[iwona][ss][sans][kurier][default]
\usetypescript
   [modern][ec]
\setupbodyfont
   [ss,10pt]
\mainlanguage
   [cz]
\setuppapersize
   [A4][A4]
\setupcolors
   [state=start]
\setuppagenumbering
   [location={footer,middle}]

\setuphead
   [chapter]
   [header=chapter,page=no,command=\grave\iwona,number=no]

%%% Odstavce %%%
\setupindenting[yes, small,first]

%%% Záložky pdf %%%
\placebookmarks[chapter,section,subsection][force=yes]
\setupinteractionscreen[option=bookmark]
\setupinteraction[state=start,
  title={Sokolí hnízdo},
  author={Mikolas Strajt}]
\setuplist[chapter][interaction=text,color=indigo]

%%% COLORS %%%
\definecolor
   [indigo]
   [r=.0, g=.0, b=.5]
\definecolor
   [grave]
   [r=.5, g=.5, b=.5]


\starttext

\startstandardmakeup
{\sc \rm Mikoláš Štrajt}
\title{Sokolí hnízdo}
\stopstandardmakeup

{\it Věnováno těm kteří přispěli ke vzniku této povídky, ať už přímo nebo nepřímo. Jmenovitě pak Romanu Novosádovi za popohánění k dopsání. Ostatní (čti ti co jsem potkával na cestě do práce) se snad v povídce poznají sami.}

\emptylines[1]

Bylo zataženo a dusno. Zahřmělo. Nad městem se spustil poslední letní liják. Takový ten teplý, co člověku není zima když zmokne.

Na vodě se dělaly kruhy a detektivům skovaným ve staré nádrži na vodu mokly vlasy. Za chvíli dojde k předání.

Ruce na kohoutcích a dusno. Výstřel. Ve starém přístavišti se strhla přestřelka. Taková ta pořádná, jako v americkém filmu.

Vodní ptáci vyděšeně utíkali a detektivům skovaným ve staré nádrži na vodu šlo o život. Předání začalo.

Ještě ten večer stál Bludowský na balkóně a kouřil. Měli kancelář až v podkroví s překrásným výhledem na celé město. Bludowského ten výhled obvykle uklidňoval, ale dneska měl chuť každého hříšníka dole poplivat, i kdyby jen odhazoval obaly od sušenek na zem.

A tak tam stál, vdechoval nikotin a čerstvý vzduch a pozoroval hříšné město. Netušil, že on sám je sledován, že kdosi o jedno patro níž a dva domy dál zapisuje kolik cigaret vykouřil.

Típnul. Dneska už pátá. Naklonil se přes zábradlí a soustředěně pustil vajgl. Vajgl letěl dolů a minul květináč. Bludowský se obvykle trefil, měl z toho takovou škodolibou radost. Ale dnes se mu nedařilo ani to. Zvednul se a šel dovnitř.

Vevnitř mezitím Waldfrucht dopisoval hlášení: ,, ...výsledkem celé akce je zadržení všech podezdřelých a lehké zranění jednoho policisty. "
 
,,Prej zadržení všech podezdřelých, pche" ušklíbl se Bludowský , ,,to tam nemůžeš rovnou napsat, že ty známý firmy z radnice se z toho zase vyvlíkly a odsrali to za ně jejich poskoci?" - ,,Klid, na každého jednou dojde..." řekl klidně Waldfrucht. ,,Aby dřív nedošlo na nás!" odsekl naštěkaně Bludowský. Waldfrucht zdvihl oči od monitoru: ,,Poslyš, co s tebou je?" - ,,Ale nic, to už je celkem běžný, že se mě každej pátek snaží někdo zastřelit" odsekl ironicky Bludowský. ,,Ty potřebuješ dovolenou." - ,,Ale né..." - ,,Jo a kdy si ji naposledy měl?"

Následovalo hluboké ticho. ,,No vidíš, vlastně už svojí pracovitostí porušuješ předpisy... Vyber si dovolenou a vyraž někam hodně daleko, kde tě nebude nic rušt, já už to nějak zvládnu" Bludowský mlčel. ,,Já ti jí rovnou napíšu" nabídl se Waldfrucht. ,,Ale kam mám teď vyrazit?" - ,,Já už bych o jednom místě věděl..."

Motorák s písklavými zvuky stoupal do hor. Rozrážel úbočí kopců a zelené hvozdy. Bludowský seděl uvnitř a koukal ven. Na obzoru se klenula duha. Jako by mu říkala, že si tam užije dost klidu. Bludowský se už těšil jako dítě na prázdniny. Ve skutečnosti už toho všeho měl plný zuby.

Vlak přejel viadukt a zastavil ve stanici Mury. Bludowský vystoupil. Teď dolů pod viaduktem a nahoru okolo sokolovny. Prošel pod viaduktem. Po mostě zrovna projížděl vlak. Odbočil do Sokolské ulice a pak ji uviděl.

Na ostrohu nad silnicí se tyčila místní sokolovna. Vypadala jako nějaký hrad. Tuto skutečnost ještě umocňovalo to, že před ní byl vykopán hluboký příkop, patrně budoucí kanalizace. V příkopu se jako nevolníci krčili kopáči, podivná to směs lidu. Bludowský vzhlédl.

Sokolovna byla opravdu majestátní. Přístavby na přístavbách jí dodávaly zákoutí, která by člověk čekal spíše na vykopávkách starověkých měst. Kromě povinné hospody v ní nechyběl sál určený k pořádání mysliveckých plesů a sportovnímu vyžití místních obyvatel, místní loutkové divadlo, byt bratra předsedy, půda plná krámů pamatujících ještě Tyrše, ubytovací kapacity pro bratry a sestry z jiných žup a strašně smradlavý pánský záchod v hospodě.

Bludowsý se rozhodl že ji vyzkouší. Zaujalo ho už jméno té hospody - U Mozarta. Prošel tedy pod vývěsním štítem s lahví piva ,,Zlatuška". Vevnitř to vypadalo jako po nějakém mejdanu. Na lustru visela čísi ponožka, barman zametal střepy a na stole spala jakási podivná existence.

,,Co se tu slavilo?" zeptal se Bludowský zvědavě. Barman vzdychl: ,,Moje narozeniny". Vyhodil střepy do koše a stoupl si za pult: ,,Věřil byste, že už je mi šedesát?" Bludowský se nadechl, chtěl něco říct, ale barman ho předběhl: ,,Takový roky, celej život za sebou. A to ani nemám základní školu... Hm. A včera jsem se dozvěděl, že už jsem dědeček - tak na to se napijem - ne?" - ,,Klidně" odpověděl Bludowský a barman mu nalil něco naprosto příšerného.

,,Ta píše, co? Dávám si ji každej den a přežil jsem i Wolfganga" - ,,Kdo je Wolfgang?" zeptal se Bludowský hlasem spáleným místní pálenkou. ,,Můj jmenovec - taky Mozart. Akorát já jsem křestním." - ,,A kdo vám dal takové zvláštní jméno?" - ,,Maminka, byla učitelka zpěvu" řekl smutně Mozart a exnul další skleničku držkopalu. Bludowský si radši objednal něco na spravení chuti a pomalu pil. Barman vzpomínal: ,,Jojo, když mi bylo dvacet, to byly časy. Tady nad sokolovnou začínaly louky a v celých Murách bylo jen pár chatek z doby před válkou, žádný chalupáři jako dnes..." Bludowského bavilo poslouchat cizí vzpomínky. ,,Já tehdy bydlel na nádraží. Vždycky jsem chodil přes viadukt za holkama. Chodili jsme tenkrát k Legerovic chatce, občas nás pustili dovnitř, akorát do sklepa ne." - ,,K Legerovic chatce? Kde to je?" - ,,Dál po cestě, ta poslední chatka. Proč se ptáte?" - ,,kamarád mi ji půjčil, budu tam teď týden bydlet." - ,,Jé, to je náhoda! Tak na to si připijem!" řekl barman a už vytahoval zpod pultu držkopal. 
,,Ne, raději ne!" zalhal Bludowský ve snaze zachránit si krk.

Zaplatil, rozloučil se a vydal se na cestu.

Na konci asfaltu, na kraji louky, stála bílá chatka na kamenné podezdívce. Bludowský vytáhl svazek klíčů, za které by se nemusel stydět ani kastelán.

Odemknout okenice. Zaháčkovat je na háčky. Nejdřív zkontrolovat čerpadlo, potom zapnout elektřinu. Do sklepa v žádném případě nechodit.

Bludowský plnil tyto pokyny a připadalo mu to jako něco, co zažil nebo možná nezažil zamlada. Chata byla obřadně odemčena, zabezpečena, mohl tedy vstoupit.

Vešel dovnitř. Celá chata se dělila jednou tenkou příčkou na dvě místnosti. První byl obyvák, s kachlovými kamínky, ledničkou, sporákem a obstarožní televizí. Pak vešel do druhé půlky - do ložnice. ,,Týjo! Waldfrucht měl pravdu! Tady jsou opravdu poklady!"

Celou jednu stěnu ložnice zabírala knihovna. A byly v ní opravdové skvosty.

Všechny díly Zelenkových dobrodružství! Zobánek! A dokonce Kapitán Afrika!

Zbytek dne strávil Bludowský čtením o Obřím kamenném muži, starostech dětské hvězdy a použití zmrzliny v boji za práva chudých. A podle toho pak vypadaly i jeho sny.

\emptylines[1]

Následující ráno stál Bludowský na louce nad chatou a vdechoval plnými doušky čerstvý ranní vzduch. Krabičku cigaret bohužel neměl. Dole v údolí se rozpouštěly chuchvalce mlhy. Období mlh začalo. Podzim se hlásil ke slovu.

Když takhle stojíte na louce nad údolím, a koukáte se dolů, uvědomíte si že jste ve skutečnosti takhle maličcí a vaše problémy malicherné. Krajina, na kterou se díváte totiž přežila Napoleona, Hitlera i Stalina. Byla tu před Rakouskem a bude tu i po něm. Jinými slovy: lidské starosti jsou pomíjivé, krása přírody je však věčná.

Přesto je tu však něco, co nás po procházce nutí nutí přečíst si noviny a pozastavovat se nad křivdami spáchanými na místním obyvatelstvu.

Bludowský přestal s filozofáním. Bylo na čase jít nakoupit nějaké zásoby.

Bludowský ještě naposledy shlédl do údolí. Bylo v něm něco znepokojivého.

Ano, Sokolovna. Sokolovna, která z údolí vypadala jako nějakým hrad, tak přesně ta samá sokolovna vypadala z kopce jako papírový domeček na modelovém kolejišti. A před ní stálo něco, co tam obvykle nestává - dav, policie a z údolí už si to šinula televizní dodávka.

Bludowský sešel níž k sokolovně. Tak tak se mu podařilo proplížit okolo davu. Na dřevěném můstku přes výkop mu uhýbala jakási slečna s červeným baretem na hlavě a fotoaparátem v ruce. Vešel do hospody.

V hospodě oproti tomu bylo všechno na svém místě. Mozart za pultem, notorici za sklenicí, dědci hráli karty. Akorát u baru seděl nějaký chlápek ve větrovce, ale vypadal mírně.

Bludowský si poručil ,,jedno" a zeptal se: ,,Co se to tam venku děje?" - ,,Ále, našli poklad" odpověděl Mozart a načepoval pivo a dodal: ,,už zase..." - ,,Jaký poklad?" - ,,Ále, za války ho sem zakopali Němci, taky jsem ho zamlada hledával..."

Bludowský se napil. Chvíli bylo ticho, které pak přerušil chlápek ve větrovce: ,,A stejně nic nenašli..." - ,,Proč myslíte?"

Chlápek upil ze své sklenice. ,,Víte zajímám se o ten zdejší poklad už pěkně dlouho, aby mi bylo jasné, že si někteří nepřejí aby se našel." - ,,Proč to?" - ,,Víte, ten zdejší poklad je patrně z větší části tvořen bednami které se na konci války ztratily z archivů. Jejich nález by některé lidi nepotěšil." - ,,Proč? Vždyť už je padesát let po válce."

Chlápek se nadechl a začal.

,,Víte, tohle jsou archivy, a Německý, takže přesný. Jsou tam popsaný zpravodajský operace od začátku do konce.

Ti lidi jsou už ve většině případů pod drnem, ale ten systém, ten ne. Ten systém totiž zastarává mnohem pomaleji než lidé. Vždyť i naše společnost stojí na právním systému starého říma.\footnote{Myšlenky převzaty ze stránek richard-1.com - autor major H.}

Bludowskému se ta řeč zamlouvala. Bylo vidět že ten chlápek ve větrovce ví, o čem mluví.

,,A jak to s tim pokladem vlastně bylo?" zeptal se zvědavě.

Chlápek ve větrovce se napil a začal: ,,To bylo takhle: Byl už konec války, byl už docela zmatek. Na zdejší nádraží dorazil vlak. Byli v něm vojáci a nějaký dobře hlídaný bedny.

Von to teda nebyl žádnej transport, jen lokomotiva a dva nákladní vagóny, ale stejně.

Vojáci chtěli jet dál do vnitrozemí, ale zrovna v ten týden voda strhla předmostí viaduktu. Když pak zjistili, že opravdu neprojedou, půjčili si pro zájmy říše tři náklaďáky z přádelny, naložili ty bedny a odvezli je někam směrem k sokolovně. Když se pak večer vrátili, náklaďáky byly prázdné. Vojáci potom odjeli vlakem zpátky odkud přijeli.

Přišlo osvobození. Všichni měli najednou jiné starosti a na tu událost s bednami se zapomělo.

Až pak, deset let po válce, se tady objevil takovej mladej němec, syn jednoho z těch vojáků. Divně se všech vyptával a pak k dovršení všeho začal u sokolovny něco kopat.

Když ho pak policajti zatkli a vyslýchali, přiznal, že hledal poklad. Prý mu otec před smrtí řekl: ,Ty bedny jsme zakopali u sokolího hnízda' a umřel. A to je celý příběh."

,,To je pěkně uhozený" namítl Bludowský. - ,,Jo to si ti policajti taky mysleli. Pak se však ukázalo, že ten němec, jistej Budenmayer, je skutečně synem jednoho z těch vojáků, co se tu na konci války pohybovali. Bohužel však německá strana neuvedla, co tu dělali.

Definitivně se to potvrdilo poté, co případ převzali tajní a uvalili na něj mlčení..." řekl chlápek a napil se.

,,Znáte to, když o něco jde, policajti začnou svorně mlčet" dodal.

,,Vy asi vůbec netušíte, s kým mluvíte" řekl tiše detektiv.

,,Nemám ten poklad rád" rozčiloval se o kus dál uplně jiný detektiv. Byl to kapitán Karotka, šéf místní policie.

,,Vážně nemám ho rád. Sou s nim jen problémy. Děti lezou, kam nemaj. Protože si myslí, že tam je. Taháme je pak kvůli tomu z různejch děr a starejch baráků.

Dospělí lezou kam nemaj. Taky všude vidí poklad. A aby toho nebylo málo, ještě se mě na to furt vyptávají novináři jako vy...."

Slečna s červeným baretem a foťákem se zašklebila a její kolegyně zastavila diktafon. ,,Děkujeme za rozhovor"

Slunce mezitím vystoupalo na nejvyšší bod své dráhy. A společně s ním stoupal i Bludowský. S Bludowským stoupal i ten chlápek ve větrovce, který se mezitím představil jako Jiří König.

Stáli tam u jakési vodárny zahrabané v zemi. Z vodárny zněl takový divný zvuk, znělo to jako vážná hudba zahraná na několik kilometrů vodovodního potrubí. Nebo jste snad podobný zvuk ještě neslyšeli?

,,Čerpaj" řekl König. Bludowský se rozhlídl a prohlížel si jednu z těch cedulek, které v ochranném pásmu vodního zdroje zakazují snad uplně cokoliv.

,,A proč si myslíte, že by měl být tady?" zeptal se Bludowský. ,,Já si nemyslím, že tady je." řekl König, ,,já si jenom myslím že tu byl. Kdyby tady byl déle , všimli by si ho vodohospodáři. Nepředpokládám, že kromě cen vody jedou ještě v nějakém spiknutí"

König se kamsi zadíval, otočil se zpět na Bludowského a začal opět vyprávět: ,,Víte. Na konci války se tu opevnili tři německý vojáci. Není to nijak dobré místo na skrývání, oni ho přesto do úplného sebezničení hájili. Podle hesla {\it Mein Ehre ist die treu}.

Vystříleli je jako králíky. Ani se jich nestačili zeptat proč to tak usilovně hájili.

Vždyť kdo by do roztrhání těla hájil vodojem jedné z mnoha přádelen, co jich v pohraničí je?" Bludowský se zadíval na vodárnu. Taková opravdu patří ke každé druhé pohraniční přádelně.

Vážná hudba z vodovodního potrubí zněla dosti tesklivě, asi na památku těch tří vojáků.

Člověk si to dnes neuvědomuje. Dokonce ani u pomníčků.

Z údolí se mezitím přiblížili dva muži. Jeden starý bělovlasý, vypadal jako dědeček z reklamy. Druhý menší, mladší, v kožené bundě. Oba měli takové ty baťůžky na notebook, oba naprosto stejné.

Došli až k vodárně a jeden z nich pozdravil: ,,Guten tag, Herr König." - König mu odpověděl ,,Guten tag, Herr Landsmann." Pak si spolu povídali něco Německy, čemuž Bludowský nerozuměl.

Potom se König rozhodl, že je představí: ,,To jsou pánové Neubauer a Landsmann z Ligy pro rovné zacházení s oběťmi války. König ještě představil Bludowského, pak si potřásli rukou a pánové řekli něco, co patrně v němčině znamená ,,těší mě".

Bludowský už ty dva někde viděl...

Ano, jeden čas s ním jezdili ráno tramvají do práce. Nevěděl co si o nich má myslet. Ty baťůžky na notebooky, košile, kravaty, a ty lejstra co občas četli vypadaly tak úředně, ale ta kožená bunda by se asi panu ministrovi moc nezamlouvala.

,,A co je přivádí sem? Snad ne poklad?" zeptal se Bludowský. ,,Válka. Dostali grant na dokumentaci a rekonstrukci pomníčků z Války".

,,Aha" řekl si Bludowský. Byl rád že je potkal.

Po určité době totiž zaroste každý pomníček plevelem, zlatá písmena vyblednou, fotografie se odlepí a bronz ukradnou sběrači kovů. Na květinách si s chutí pochutnají brouci.

Pomníčky nejsou určeny k tomu, aby vydržely věčnost. Ba naopak.

Jsou určeny k tomu, aby svým chátráním říkaly: ,,Oprav mě! Oprav mě a zamysli se, proč tu stojím. Čí památku asi uctívám?"

Logickým důsledkem pak je, že se u každého pomníčku dříve či později objeví někdo, kdo ho jako stařenka na vesnickém hřbitově omete, položí nové květiny a zapálí svíčku.

Starší němec obřadně vytáhl zapalovač a podal ho svému druhovi v kožené bundě. Ten obřadně zapálil svíčku.

Tak tam stáli a mlčeli čtyři muži. Z vodárny zněla teskná hudba. Budenmayer by měl radost.

Je jedno jaké národnosti jsou oběti.

Válka je stejnej vůl pro všechny.

Toho večera Bludowský nemohl usnout. Ležel a přemítal zatímco se dole na stráni pod chatou řítil už asi pátý vlak. Přemítal o válce.

Jasně, viděl už mrtvolu a potkal pár vrahů. Pracovně samozřejmě. Ale to bylo něco jiného.

Taková válka, to je něco mnohem horšího. Něco, co kvasí napříč celou společností. Něco, co nezabíjí konkrétní jména.

Bludowský byl rád, že prožil celý život v míru. Válku nechtěl. Přesto ho však zajímal ten pocit. Ta vůně války ve vzduchu.

Když má někdo před spaním takové černé myšlenky, obvykle to nedopadne dobře. Ve snu se mu pak míhají defilé šílených kudrnatých básníku, dlouhonohých dívek s červenými barety, vojenských transportů, husarů na koních a tří nákladních automobilů.

S tím rozdílem, že ta tři nákladní auta byla skutečná.

\emptylines[1]

Po padesáti letech jely opět tři náklaďáky z Mur okolo Sokolovny do lesa. Tentokrát však chtěly poklad vykopan. Jeden z těch náklaďáků byl totiž vrtná souprava.

A Bludowského jeho hlučný průjezd pod okny spolehlivě probudil. Chvíli se se slovy ,,Ježišmarjá" vzpamatovával ze snů, ale pak už vyrazil.

Vyrazil z kopce dolů, do města. U sokolovny nebyl takový dav jako včera. Akorát ta slečna s červeným baretem dělala nějaké fotografie.

Kopáče ve výkopu vystřídali archeologové, a to je snad ještě šílenější druh lidí. Nejhorší na tom je, že jsou vlastně všichni spolužáci, protože ta škola, kde se člověk vyučí archeologem, je obvykle v celém státě jen jedna. ,,Starej Indy Jones? Jó, toho znám, ten studoval s naším profesorem egyptologie" odpoví vám na vaše nejzavilejší dotazy.

V hospodě bylo ještě zavřeno a tak detektiv pokračoval dál do údolí. Měl domluveno s Kōnigem, že ho navštíví.

,,Můj dům poznáš, vypadá poněkud nepatřičně" řekl mu včera Kōnig. A měl pravdu.

Desetimetrová plachetnice opravdu nepatří mezi věci běžně vídané před roubenkami na úbočí hor.

Bludowský zazvonil a Kōnig mu šel otevřít. ,,Děláte lodě?" zeptal se detektiv. - ,,Lodě. A nejen je. Stoly, židle, všechno co je ze dřeva. Od kolébky po rakev" - ,,To máte dobrý, já jsem většinou potřeba jen u těch rakví" povzdechl si detektiv.

Vešli dovnitř. Kōnig detektiva uvítal, nabídl mu čaj, omluvil se mu za nepořádek, smetaje přitom tlustou vrstvu pilin ze židle.

,,Víte, trochu jsem o tom pokladu včera přemýšlel." začal Bludowský. - ,,A na co jste přišel?" zněla otázka z kuchyně od vařícího čaje. - ,,Napadlo mě. Co když je ten poklad jen další hloubková lež?

Taková myšlenka, o které víte, že to není pravda, ale říkáte ji dál, protože jí všichni rozumí.

Něco jako ,Lidé s brýlemi vypadají chytře' nebo ,Dřív si lidé mysleli, že je země placatá' čí ,Kapitalismus je tu odjakživa'...

Chtěl jsem jenom říct: Co když je to všechno jinak? Co když je to jenom pár beden zvlhlého papíru? Co když to není zakopáno, ale třeba to v klidu leží někde na půdě? Proč se za tím pachtit, cožpak nemáme dost starých lejster v archivech?"

Kōnig přišel s čajem a zamyšleným výrazem ve tváři. Chvíli převaloval myšlenku v hlavě a pak odpověděl: ,,Víte proč to ti vojáci tehdá skovávali? Jeden by skoro řekl, aby to nikdo nenašel, že?

Ale to není pravda. To je - jak byste řekl vy - hloubková lež. Oni to totiž skovávali, aby to někdo našel...

Teda ne jejich nepřátelé, spíš jejich potomci."  Pak odešel do sousedního pokoje a vrátil se s fotografií v ruce: ,,To je pohled na město za války. Půjčte si ho. Až budete mít čas, můžete se po tom pokladu podívat." - Bludowský si vzal fotografii a zadíval se na ní zblízka: ,,Jé, tady je ta chata co bydlim. Koukám, že se od války moc nezměnila."

,,Od války se nezměnilo překvapivě hodně věcí" povzdechl si tesař.

Potom se Bludowský rozloučil se slovy ,,nebudu už zdržovat"\footnote{Kōnig, ačkoliv dobře věděl, že ho zdržuje, mu na to říkal: ,,ale ne, vůbec nezdržujete"} a vydal se na cestu. Proti němu jel z kopce traktor Samizdat. Podle takovýhle ,,u strejdy v garáži na koleně" vyrobených traktorů se venkov pozná spolehlivě. Stoupal dál. Po náspu železniční tratě se zřejmě plazil nějaký stroj. Hluboko vyryté stopy naznačovaly, že se někdo, kdo nebral ohled na překážky, pokoušel dostat přes násep. Bludowský je zkusil. Vedly přímo nahoru na násep a pak přímo dolu. Někdo spěchal dokonce tolik, že přestříhal drátovod!

Možná by teď bylo dobré pro neznalé vysvětlit, co to takový drátovod je. Drátovody sloužily železničářům ke stahování závor a ovládání návěstidel. V podstatě se jednalo o dlouhé dráty natažené z hradla, kde byl železničář, k návěstidlu, kde už díky tomu být nemusel. Tyhle dráty vedly po takových malých sloupcích podél trati. A na některých lokálkách je můžete vidět dodnes.


\stoptext 