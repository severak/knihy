\setupoutput
   [pdftex]
\enableregime
   [utf]
\definetypeface[iwona][ss][sans][kurier][default]
\usetypescript
   [modern][ec]
\setupbodyfont
   [ss,10pt]
\mainlanguage
   [cz]
\setuppapersize
   [A4][A4]
\setupcolors
   [state=start]
\setuppagenumbering
   [location={footer,middle}]

\setuphead
   [chapter]
   [header=chapter,page=no,command=\grave\iwona,number=no]

%%% Odstavce %%%
\setupindenting[yes, small,first]

%%% Záložky pdf %%%
\placebookmarks[chapter,section,subsection][force=yes]
\setupinteractionscreen[option=bookmark]
\setupinteraction[state=start,
  title={Sokolí hnízdo},
  author={Mikolas Strajt}]
\setuplist[chapter][interaction=text,color=indigo]

%%% COLORS %%%
\definecolor
   [indigo]
   [r=.0, g=.0, b=.5]
\definecolor
   [grave]
   [r=.5, g=.5, b=.5]


\starttext

\startstandardmakeup
{\sc \rm Mikoláš Štrajt}
\title{Sokolí hnízdo}
\stopstandardmakeup

{\it Věnováno těm kteří přispěli ke vzniku této povídky, ať už přímo nebo nepřímo. Jmenovitě pak Romanu Novosádovi za popohánění k dopsání.}

Bylo zataženo a dusno. Zahřmělo. Nad městem se spustil poslední letní liják. Takový ten teplý, co člověku není zima když zmokne.

Na vodě se dělaly kruhy a detektivům skovaným ve staré nádrži na vodu mokly vlasy. Za chvíli dojde k předání.

Ruce na kohoutcích a dusno. Výstřel. Ve starém přístavišti se strhla přestřelka. Taková ta pořádná, jako v americkém filmu.

Vodní ptáci vyděšeně utíkali a detektivům skovaným ve staré nádrži na vodu šlo o život. Předání začalo.

Ještě ten večer stál Bludowský na balkóně a kouřil. Měli kancelář až v podkroví s překrásným výhledem na celé město. Bludowského ten výhled obvykle uklidňoval, ale dneska měl chuť každého hříšníka dole poplivat, i kdyby jen odhazoval obaly od sušenek na zem.

A tak tam stál, vdechoval nikotin ačerstvý vzduch a pozoroval hříšné město. Netušil, že on sám je sledován, že kdosi o jedno patro níž a dva domy dál zapisuje kolik cigaret vykouřil.

Típnul. Dneska už pátá. Naklonil se přes zábradlí a soustředěně pustil vajgl. Vajgl letěl dolů a minul květináč. Bludowský se obvykle trefil, měl z toho takovou škodolibou radost. Ale dnes se mu nedařilo ani to. Zvednul se a šel dovnitř.

Vevnitř mezitím Waldfrucht dopisoval hlášení: ,, ...výsledkem celé akce je zadržení všech podezdřelých a lehké zranění jednoho policisty. "
 
,,Prej zadržení všech podezdřelých, pche" ušklíbl se Bludowský , ,,to tam nemůžeš rovnou napsat, že ty známý firmy z radnice se z toho zase vyvlíkly a odsrali to za ně jejich poskoci?" - ,,Klid, na každého jednou dojde..." řekl klidně Waldfrucht. ,,Aby dřív nedošlo na nás!" odsekl naštěkaně Bludowský. Waldfrucht zdvihl oči od monitoru: ,,Poslyš, co s tebou je?" - ,,Ale nic, to už je celkem běžný, že se mě každej pátek snaží někdo zastřelit" odsekl ironicky Bludowský. ,,Ty potřebuješ dovolenou." - ,,Ale né..." - ,,Jo a kdy si ji naposledy měl?"

Následovalo hluboké ticho. ,,No vidíš, vlastně už svojí pracovitostí porušuješ předpisy... Vyber si dovolenou a vyraž někam hodně daleko, kde tě nebude nic rušt, já už to nějak zvládnu" Bludowský mlčel. ,,Já ti jí rovnou napíšu" nabídl se Waldfrucht. ,,Ale kam mám teď vyrazit?" - ,,Já už bych o jednom místě věděl..."

Motorák s písklavými zvuky stoupal do hor. Rozrážel úbočí kopců a zelené hvozdy. Bludowský seděl uvnitř a koukal ven. Na obzoru se klenula duha. Jako by mu říkala, že si tam užije dost klidu. Bludowský se už těšil jako dítě na prázdniny. Ve skutečnosti už toho všeho měl plný zuby.

Vlak přejel viadukt a zastavil ve stanici Mury. Bludowský vystoupil. Teď dolů pod viaduktem a nahoru okolo sokolovny. Prošel pod viaduktem. Po mostě zrovna projížděl vlak. Odbočil do Sokolské ulice a pak ji uviděl.

Na ostrohu nad silnicí se tyčila místní sokolovna. Vypadala jako nějaký hrad. Tuto skutečnost ještě umocňovalo to, že před ní byl vykopán hluboký příkop, patrně budoucí kanalizace. V příkopu se jako nevolníci krčili kopáči, podivná to směs lidu. Bludowský vzhlédl.

Sokolovna byla opravdu majestátní. Přístavby na přístavbách jí dodávaly zákoutí, která by člověk čekal spíše na vykopávkách starověkých měst. Kromě povinné hospody v ní nechyběl sál určený k pořádání mysliveckých plesů.

\stoptext 