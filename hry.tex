\setupoutput
   [pdftex]
\enableregime
   [utf]
\usetypescript
   [modern][ec]
\setupbodyfont
   [ss,10pt]
\mainlanguage
   [cz]
\setuppapersize
   [A4][A4]
\setupcolors
   [state=start]
\setuppagenumbering
   [location={footer,middle}]

\setuphead
   [chapter]
   [header=chapter,page=no,command=\indigo,number=no]

%%% Odstavce %%%
\setupindenting[yes, small,first]

%%% Záložky pdf %%%
\placebookmarks[chapter,section,subsection][force=yes]
\setupinteractionscreen[option=bookmark]
\setupinteraction[state=start,
  title={Hry s ohněm a jiné povídky},
  author={Mikolas Strajt}]
\setuplist[chapter][interaction=text,color=indigo]

%%% COLORS %%%
\definecolor
   [indigo]
   [r=.0, g=.0, b=.5]


\starttext

\startstandardmakeup
Mikoláš Štrajt
\title{Hry s ohněm a jiné povídky}
\placecontent
\page
\stopstandardmakeup

\chapter{Slunečné odpoledne mladého inženýra}

Bylo krásné odpoledne. Slunce svítilo, nebe bez mráčku. Mladý inženýr stál u lavičky v parku a díval se na nebe. Vedle něj stála laborantka Taťána a dívala se střídavě na nebe a na hodinky.

„Už letí!“ vykřikla. Inženýr se podíval do nebe. Ano, u slunce byla opravdu nějaká černá skvrnka. „Už letí, už letí, můj táta se po třech letech vrací z Jupitera!“ jásala laborantka a samou radostí objala inženýra. „Počkej, ten bude mít radost až mu budu vyprávět, co všechno se za ty tři roky stalo.“ Černá skvrnka se zvětšila a s pištivým bzučením proletěla nad nimy. Laborantka se ještě podívala na její odlet. Chvíli tam stáli ruku v ruce, pozorovali vědecko-technický pokrok zosobněný jejím otcem. Pak jí vyprovodil k autobusové zastávce linky ke kosmodromu a znova se posadil na lavičku.

Ptačí zpěv ustal. Inženýr seděl na slunci a přemýšlel o tom, jaký má krásný život. Myslel na Taťánu, a radoval se z toho, že žije zrovna v Inženýrsku, které je proslulé svými zahradami široko daleko. Seděl na sluníčku a vyhříval se.

V tom ho vyrušilo vrzání vozíku a čísi hlas.„To si někdo žije. Sedí si na lavičce a my aby jsme tu běhali po všech sadech a zahradách, co jich tu je“ Otočil se, a uviděl že jsou to dva zahradníci s vozíkem. Došli k němu. Starší zahradník řekl: „Dovolíte?“ a stoupl si na lavičku. Mladší mu podal šroubovák a baterky. Starší rozhrnul větve stromu za lavičkou a mladší mu je podržel. Na stromě byla ptačí budka, kterou starší zahradník rozšrouboval. Vyměnil v ní baterky a zase ji zašrouboval. Pak stiskl bidýlko a parkem se zase rozléhal ptačí zpěv. posadili se na lavičku a starší zahradník si zapálil.

Inženýr na ně nevěřícně koukal a pak se odhodlal k otázce: „To mi jako chcete říct, že tu v parku nejsou živí ptáci?“ Mladší se na něj podíval jako na debila a řekl: „Jo, je to tak. Před rokem všichni vypověděli smlouvu“ Starší zahradník potáhl a dodal: „Tak, tak“

FINIS

\page

\chapter{Jůlie}

Julie opustila Město. Opustila Město, protože mnoho jeho ulic, náměstí a nábřeží bylo zkompromitováno nějakou špatnou událostí. Jako když jedete autem po dálnici, posloucháte muzikál, a když pak jedete tu samou cestu zpět, muzikál se vám vrací. Julie vlastně neměla moc důvodů proč ve Městě zůstávat.

Těmito slovy zahájil svou řeč průvodce v autobuse s nápisem City sightseeing.

Tato prohlídka je o skutečných emocích, o skutečných lidských životech. O takové tváři města, kterou po dlouhou dobu znala jen samotná Julie.

Autobus vyrazil a turisté všech ras a národností zpozorněli. Autobus vjel do obyčejné všední uličky. V této ulici se Julie rozešla se svou první láskou. Brečela. A na tomto přechodu ji pak málem přejela tramvaj.

Turisté vyfotili všecko. Šedou ulici, přechod i tramvaj. Vypadalo to jak jedno velké divadlo, způsob jak prodat obyčejnou šedou ulici.

Julie však byla krásná a chytrá, takže nezůstávala dlouho sama. Na nádraží, které právě vidíte na levé straně, se loučila se svým tehdejším přítelem. Tahle scéna se později objevila ve filmu Past na kočku.

Turisté si vyfotili nádraží a když otočili své zvědavé krky zpět do autobusu, průvodce pustil filmovou ukázku. Autobus vjel do úplně obyčejné ulice. V tomto červeném domě prožila Julie své mládí. Do tohoto parku naproti si jako dítě chodila hrát. Celé to vypadalo jako nějaká habaďůra. Jako způsob, jak turistům prodat něčí obyčejný život. Jako by turisté něměli své vlastní životy.

Autobus zahnul do ulice, kde měl být poslední krvavý šrám na mapě města.

Na tomto hřbitově jsou pochováni někteří Jůliini příbuzní. Její strýc Max a teta Žofie, která byla pro malou Jůlii opora.

Autobus objel hřbitov a vrátil se na místo, kde výlet začal.

Děkuji vám, že jste se s námi projeli na této úchvatné prohlídce plné skutečných emocí. Těšíme se na vaši další návštěvu. Má-li někdo nějaký dotaz, ptejte se.

Zvedla se jedna žena.

,,Jak jsi jen mohl?'' a vrazila průvodcovi pořádnou facku. Odešla. Byla to Julie, chytrá a krásná.

\hairline

Věnováno všem Jůliím, Rómeům a průvodcům City sightseeing.

\page

\chapter{Hry s ohněm}
Byl krásný jarní večer. Bludowský seděl a kochal se výhledem na město zalité zlatou září zapadajícího slunce. Odkudsi z dálky stoupal dým. Něco tam hořelo. ,,Aby to tak nebylo pro nás...'' řekl Bludowský.

A bylo to pro něj. To co hořelo byla totiž místostarostčina garáž a jak volala požární inspekce, náhoda to nebyla.

Nazítří ráno kráčel Bludowský na radnici. Namířil si to do vrátnice. Ve vrátnici seděl Kostěj Nesmrtelný, říkalo se mu tak proto, že jako jediný vydržel na radnici skoro padesát let. Střídali se starostové, politické strany i režimy, jenom Kostěj zůstával. ,,Potřeboval bych mluvit s paní místostarostkou'' řekl Bludowský. ,,Třetí patro, dveře 32. A copak paní místostarostce chcete?'' dotíral Kostěj. ,,To vás nemusí zajímat''

,,Dále!'' ozvalo se a Bludowský vstoupil. Za stolem seděla místostarostka - Ludmila Dobrovská. Byla to zástupkyně opozice - Strany za lepší přechody. ,,Kriminální policie,'' představil se Bludowský ,,potřeboval bych se k vám podívat na tu garáž - kdy by se vám to hodilo?'' - ,,Klidně dnes večer, přijďte třeba v sedm, já mám bohužel přes den moc práce, to víte, volby se blíží. Chtěl jste ještě něco?'' - ,,No jenom se zeptat, co si o tom myslíte.'' - ,,Těžko říct. Když začalo hořet, nebyla jsem doma. Hasiče volala moje dcera. A žádná podezdření nemám.''

Večer Bludowský skutečně přišel. Místostarostka bydlela v jednom z těch nových domů na kraji města. tento dům byl stranou od těch ostatních novostaveb, hezky v poli s výhledem na řeku. U cesty byla zmíněná garáž, samotný dům byl asi o deset metrů dál v pozemku. Bludowský zazvonil.

,,Nejprve bych se s dovolením podíval na místo požáru a pak, kdybyste neměla nic proti, bych si s vámi trochu promluvil.'' - ,,Jistě. Jak si přejete.'' řekla místostarostka a Bludowský strhl pečeť z dveří garáže a vešel. Uvnitř byla tma. Bludowský si nejprve posvítil na řádně očouzelý a ohořelý strop. Pak na podlahu, tam leželo něco, co bylo kdysi patrně kanystrem benzínu. U stěn u trosek dřevěných poliček stála sekačka. A uprostřed místnosti podivná věc - ohořelé křeslo. ,,To je vaše?'' - ,,Ne.''

Po ohledání místa činu detektiv přistoupil k výslechu svědků. V době požáru byla doma pouze Valérie, místostarostčina dcera. Místostarostka v té době vezla svého syna Nikolase z hodiny kytary v základní umělecké škole a její manžel, povoláním pilot dopravního letadla, byl na služební cestě v zahraničí.

Valérie byla typická studentka umělecké školy. Podivný vzhled, nosánek trochu nahoru. Bludowský zahájil výslech: ,,Jak jste si všimla, že hoří?'' - ,,Začal zvonit požární hlásič.'' - ,,A pak?'' - ,,Pak jsem běžela ke garáži, chtěla jsem to uhasit, ale hořelo už moc'' - ,,Nevšimla jste si něčeho zvláštního?'' - ,,Ne.'' - ,,Byla garáž odemčená nebo otevřená?'' - ,,Jo, byly otevřený dveře na zahradu.'' - ,,A kdy jste volala hasiče?'' - ,,Hned potom, běžela jsem dovnitř a zavolala je.'' - ,,Dobrá, ještě poslední otázku: Kdo myslíte že to udělal?'' - ,,Myslím, že je za tím nějaká politika.'' Její odpověď se nazítří potvrdila.

Bludowský spěchal to ráno rovnou na radnici. Před radnicí stáli jeden novinář, tiskový mluvčí radnice, Kostěj Nesmrtelný a dva strážníci. Předmět jejich zájmu byl prostý - vytlučené sklo vstupních dveří a nad tím nasprejovaný nápis: To máte za Dobrovskou!

Uvnitř ležela na podlaze skleněná láhev s ohořelým kapesníkem uvnitř, zřejmě se někdo pokusil zapálit radnici Molotovem, ale nepovedlo se mu to.

,,Pojďte za mnou pane inspektore'' řekl Kostěj ,,toho darebáka natočila kamera, já vám to pustim.'' Bludowský vešel do Kostějova kamrlíku. Bylo tam několik televizních obrazovek, jedna z nich ukazovala potemnělý parčík u radnice, v rohu byl napsán čas 1:25. Kostěj stiskl tlačítko PLAY: Před radnici přijela motorka. Maskovaný řidič se rozhlédl a odstavil motorku. Vytáhl sprej a napsal nápis. Pak rozbil kamenem sklo a hodil dovnitř Molotov. V tu chvíli ho něco vylekalo, tak sedl na motorku a odfrčel.

,,Hmmmm. Tak takhle jsem si to taky představoval. Vraťte to! ... Má zakrytou poznávací značku.'' řekl detektiv. ,,Myslel jsem, že vám to pomůže'' řekl Kostěj. ,,Tak mi řekněte, co si o tom myslíte, kdo a proč to udělal?'' - ,,Ale pane detektive, do toho já se nepletu, to je politika!'' odpověděl pohoršeně Kostěj.

Ačkoliv o tom skandálu očividně mluvili všichni na radnici, nikdo se o něm nechtěl s detektivem bavit. Starosta s místostarostkou byli kdesi na otvírání jakési stavby. Jediný, kdo neodmítl vypovídat, byl tiskový mluvčí. Bylo to částečně proto, že jeho povoláním bylo odpovídat na dotěrné otázky, částečně proto, že měl už pokyny shora: ,,Nebudu vám zastírat, že tady jsou jisté politické spory. Ale jak už jsem řekl novinářům, tohle je radnice, my domy nezapalujem, my nejsme mafie!'' Ale nemáte do ní zas tak daleko, pomyslel si Bludowský.

Toho večera čekal Bludowský v restauraci s Waldfruchtem. Waldfruch byl nadřízený a zároveň kolega Bludowského. Seděl, pil a stěžoval si: ,,Je to hrozný, všichni dělaj problémy! Mě dělaj problémy squateři, tobě zase politici. Dokonce se bouřej i odbory trolejbusáků. Hrůza. Jestli se v tomhle městě něco semele, tak to bude veselo....''

Waldfruch vzdychl a napil se. V tu chvíli přišla ta, na kterou čekali, Waldfruchtova přítelkyně Bety.

Bety byla hezká i chytrá. Pracovala jako sekretářka vedoucího odboru městského investora, fakticky však díky svým schopnostem řídila odbor sama. Waldfrucht se s ní seznámil při jedné ze svých četných pracovních návštěv radnice.

,,Omlouvám se, ale měli jsme nějaké problémy na stavbě'' omluvila svůj pozdní příchod Bety a přivítala se s Waldfruchtem.

,,Potřeboval bych tvou pomoc,'' oslovil ji Bludowský ,,úplně nechápu, kdo tam od vás by mohl mít něco proti Dobrovské''.

Bety se posadila a řekla: ,,Tak ti zopakuju politickou situaci hezky od začátku.'' Bety jim už několikrát pomohla, jednou už se pod ní povážlivě houpala židle, ale její šéf se za ní přimluvil. Měla však výhodu, protože to, co říkala, byla pravda dohledatelná z listin a zápsů jednání zastupitelstva. Napila se a nadechla, vysvětlit politickou situaci byl těžký úkol.

,,No, v podstatě se dá město politicky rozdělit podle řeky Akvy. Ti, co bydlí na západ od Akvy, volí Černé, a ti, co bydlí na východ, volí Červené. No a protože je zapádní část města větší, vyhrával posledních deset let Kozák s Černými. Jenže posledné volby toho voliči měli dost, tak přišla změna. Červení i Černí dostali stejně málo hlasů a zbytek se rozdělil právě mezi stranu Dobrovské a Zelené. Obě strany zůstaly neutrální, takže Kozák s Taškárem se je před každým hlasováním snaží přetáhnout na svou stranu.'' - ,,Takže Dobrovská je pokaždé jejich naděje na schválení?'' - ,,Dá se to tak říct.'' - ,,A o čem se právě hlasuje?''

,,To je právě to!'' vzdychla Bety. ,,Už dlouho se o ničem nehlasovalo. Poslední věc, kterou politic udělali je, že otevřeli most dálničního obchvatu.'' - ,,Hmmmm, myslim, že bych se s těmi lidmi měl setkat a promluvit si s nimi'' - ,,Zkusim to zařídit.''

Následující den při návratu z oběda měl Bludowský dobrou náladu. Bety se podařilo zařídit pro něj propustku na jakousi vernisáž mladých umělců podporovanou radnicí, prý tam bude celá politická reprezentace.

Bludowský vešel do budovy policie. U recepce se nějaký muž hádal se strážníky: ,,Neodejdu dokud si nepromluvim s nějakým detektivem!'' - ,,Jestli už konečně neodejdete dám vás zatknout!'' křičel vztekle strážník.

,,Já si s ním promluvím.'' řekl Bludowský ,,Pojďte! Jsem detektiv.'' Muž ho následoval do kanceláře. Posadili se, Bludowský nabídl kávu, muž ji odmítl. Bludowský začal výslech: ,,Tak jak se jmenujete?'' - ,,Viktor Wechsel, směnárník, teda byl jsem...'' - ,,No a co vám kdo udělal?'' - ,,Ukradli mi křeslo.'' - ,,Křeslo?'' podivil se Bludowský.

,,Křeslo, po tetičce Žofii, byla to rodinná památka. Už o tom víte týden a nikdo s tim nic nedělá!'' odpověděl Viktor. Bludowský vzdychl: ,,A jak to křeslo vypadalo?'' - ,,Takhle!'' řekl Viktor a vstal ze židle a šel k nástěnce ,,jako tohle křeslo, co tu máte vyfocený. Kdyby nebylo ohořelý, byl bych přísahal, že je moje.''

Bludowský zpozorněl, fotografie ohořelého křesla na spáleništi byla z garáže místostarostky. ,,Znáte místostarostku?'' zeptal se Bludowský. ,,Co s tim má co společnýho místostarostka? Vždyť mě ukradli křeslo!'' rozčiloval se Viktor. Tudy cesta nevedla.

Bludowský vzdychl, sepsal s Viktorem protokol a svatosvatě mu slíbil, že se po jeho křesle podívá. Pak se pustil do čtení politických analýz a četl je po zbytek odpoledne.

Večer Bludowský oblékl oblek a vyrazil na vernisáž. Na vernisáži byli mladí umělci, jejich učitelé, kritici a politická reprezentace. Byla tam Dobrovská s manželem, bavila se s nějakým učitelem. Byla tam i její dcera, s přítelem, byl to nějaký motorkář nebo tak něco, bavili se s nějakou podobnou umělkyní. Byl tam Waldfrucht, vedli se s Bety za ruce a prohlíželi si obrazy. Taškár, vůdce Červených si taky se zaujetím prohlížel obrazy, kdysi býval malířem (pokojů) než se stal politikem. A starosta Kozák stál u vchodu a komandoval nějakého podržtašku.

Bludowský si prohlédl výstavu. Vypadala podivně a nelogicky, inu mladí umělci. Mladá Dobrovská fotografovala. Dělal fotografie divných lidí v divných situacích. Bludowskému se moc nezamlouvaly.

Bludowský přistoupil ke starostovi: ,,Dobrý večer, pane starosto.'' - ,,Co tady děláte?'' vyštěkl starosta. ,,Ále, taky podporuju mládé umělce'' řekl ironicky detektiv. ,,Nechte si ten tón! Přišel jste vyšetřovat nebo se jen dívat?'' - ,,No kdyby vám to nevadilo, tak bych vám pár otázek dal...'' - ,,Hmmm. Tak ale pojďme někam stranou, je tu plno novinářů.''

Vyšli ven, zašli za roh do takové tiché ulice a starosta si zapálil. ,,Ptejte se!'' řekl nekompromisně. ,,Jak se vám spolupracuje s Dobrovskou?'' - ,,Je tvrdohlavá, má vlastní názory a nenechá se moc přesvědčit. Ale já, ani nikdo z Černých jí přece garáž zapalovat nepůjdeme'' - ,,A kdy nakonec hlasovala proti vám?'' - ,,Když se probíralo, zda se má zvýšit poplatek za svoz odpadu v chatové osadě Břízka'' řekl potutelně starosta. ,,Ale, pane starosto, nedělejte ze mě blbce. Myslím něco podstatného, třeba kauzu Logistický areál.'' - ,,Logistický areál? To bylo před měsícem a nakonec se to stejně schválilo i přes odpor Dobrovské. Pane detektive, zastupitelstvo už skoro dva týdny nezasedalo, v čem by Dobrovská vadila?'' zeptal se starosta. Bludowský se nadechl, ale v tu chvíli starostovi zazvonil mobil a starosta odešel zuřivě telefonovat.

Bludowský se vrátil do sálu, Taškár mezitím odešel, tak neměl koho vyslýchat. Sedl si na bar a po chvíli se dal do diskuze s jedním vysokým mladíkem. Za chvíli se jejich hovor stočil na dceru Dobrovské.

,,Valérie má talent,'' začal dlouhán ,,akorát mě trochu překvapuje, že sem nedala svoje nejlepší věci, třeba ty fotky z pískovny nebo hořící křeslo.'' - ,,Hořící křeslo?'' - ,,Ano, hořící křeslo. Zatim to byl jen nápad, ale skvělej. Muž v obleku sedící v kanceláři na hořícím křesle. Původně to měla být fotomontáž a měl mít Kozákovu tvář, ale to už bylo přece jenom trochu moc.'' řekl pobaveně dlouhán. Bludowský se tomu zasmál a dál už se jejich řeči týkaly jen umění.

Večírek pokračoval. Starosta s místostarostkou vysvětlovali novinářům, proč je důležité podporovat umělce. Mladá Dobrovská se mezitim na parketu svíjela okolo nějakého nagelovaného frajírka. Najednou se strhla bitka.

Frajírek seděl na zemi a snažil se zastavit krev, která mu kapala z nosu na drahou košili. ,,Nebudeš se na ní takhle lepit! Je to moje holka! Si nemysli, že když ti tvůj fotřík koupí k narozeninám nový bávo, tak že můžeš dovolit balit všechny holky! A s tebou jsem taky skončil!'' zařval motorkář a naštvaně odešel. ,,Lukáši, počkej!'' řekla zoufale mladá Dobrovská a utíkal za ním.

Místostarostka mezitim taktně naznačila manželovi, ať mlčí. Poslední, co před volbama potřebovala, bylo, aby se její dcera objevila v bulváru. Pro ni večírek skončil fiaskem.

Bludowský jel domů tramvají, cestou přemýšlel a usnul. Tak se musel vracet z konečné. Nakonec ho něco napadlo.

Zítra si Bludowský počíhal na Valérii Dobrovskou před školou: ,,Potřeboval bych s váma mluvit.'' - ,,Jestli se se mnou chcete bavit o tom, co provedl můj přítel na vernisáži, tak vám nic neřeknu.'' - ,,Nezajímá mě váš přítel, mě zajímá umění. Proč jste nevystavila tu fotografii hořícho křesla?'' - ,,Protože jsem ji nestihla udělat.'' - ,,Ale zkoušela jste to.''

,,Jak tohle víte?'' podivila se Valérie Dobrovská. ,,To ohořelé křeslo ve vaší garáži...'' odpověděl detektiv.

Valérie mlčela. ,,Nemá smysl mlčet.'' řekl klidně detektiv. Valérie sklopila oči k zemi a pak se podívala na detektiva. ,,Máte pravdu,'' začala ,,zkoušela jsem to vyfotit. Abych nějak nezničila podlahu, dělal jsem to v garáži, protože je tam beton. To křeslo nechtělo hořet, tak jsme ho trochu polili benzínem, ale asi jsme to přehnali...'' - ,,A ten nápis na radnici?'' - ,,To napadlo Lukáše udělat z toho politiku. Říkal, že se přijde na to, že to chytlo od benzínu, tak pak jel na radnici a udělal to'' šeptala Valérie. Oba mlčeli. Valérii vytryskly slzy.

,,Prosím, pane detektive, zkuste to nějak ututlat. Mámě to zničí kariéru a zničí to i mě...'' začala plakat. Bludowský otevřel pusu, chtěl ji pokárat, ale zarazil se. Uvědomil si, že by tím způsobil víc škody než užitku. Nadechl se, zapálil si a přemýšlel. Valérie tiše vzlykala.

,,Víte co, já to ututlám, ale něco za to chci...''

Asi za týden pomáhal Bludowský Valérii vykládat křeslo. Zazvonil na zvonek. Vyšel Viktor Weschel a překvapeně zíral na křeslo.

,,Je to jen replika. To vaše shořelo před čtrnácti dny. Ti, kdo se zasloužili o jeho likvidaci se složili na nové.'' vysvětlil Bludowský a otočil se na Valérii.

Bylo jaro. Ptáci zpívali, Valérie s Lukášem se dali zase dohromady a volby se neúprosně blížili.

EOF

\page

\chapter{Bludowský slaví Vánoce}

Na Hlubočici padnul sníh. Byla tma a silný vítr hnal sníh z polí.

\startlines
Ó lokomotivo jenž osvobozuješ. Již slyším kovovou píseň kolejí.
Ó lokomotivo mocná, tvá světla jak oči šelmy září tmou.
Ó lokomotivo silná, již slyším známý takt pražců a vrnění dieselova motoru.
Ó lokomotivo krásná, již vidím tě prorážet sněhovou závěj.
Ó lokomotivo čistá, tvá píšťala zní jak andělské troubení.
Ó lokomotivo zrádná, tvé brzdy mne již nemohou zastavit.
Ó lokomotivo milosrdná, ani to moc nebolelo...
\stoplines

A pak již jen desítky párů kol a Cháron v železniční uniformě.

Strojvedoucí zavřel regulátor, vysílačkou vyděšeně uvedl v chod mašinerii starající se o zprovoznění tratě, a pak se předpisově zhroutil.

{\em Tichá noc, svatá noc.}

Na druhém konci města to vypadalo předpisově idilicky. Světelné řetězy, vánoční stromečky,vůně skořice a cukroví a poctivý slaměný betlém. A vzduchem se nesla Tichá noc. U stánku balící služby stáli dva muži. Byli to detektivové z kriminálky. I oni hledali v předvánočním schonu trochu klidu a míru. Právě jim skončila služba.

,,Nemám rád Vánoce'' vzdychl Bludowský. ,,Proč to?" podivil se Waldfrucht. ,,Osamělí jsou ještě osamělejší, chudí chudší, podvodníci vykutálenější a reklamy stále dotěrnější..." řekl Bludowský a podíval se na Santu Klause v protější výloze a smutně dodal: ,,Z Vánoc se staly svátky konzumu..." - ,,Třeba si to jen moc bereš" - ,,Neberu, podívej se na statistiku sebevražd..." odpověděl Bludowský pochmurně.

{\em Dej Bůh štěstí tomu domu, my zpíváme víme komu...}

Přes pult se naklonil brigádník a podával Waldfruchtovi těžký nazdobený balíček.. ,,Myslíš si, že se jí to bude líbit?" - ,,Určitě" Vyšli ven. Waldfrucht hrdě nesl dárek.

Před obchoďákem nějaká žena lovila zmrzlými prsty drobné z peněženky aby je mohla naházet do automatu na jízdenky. Přitočil se k ní nějaký muž, snad potřeboval rozměnit. Peněženka spadla do sněhu a už s ní utíkal pryč. ,,Zloděj! Chyťte ho!" Detektivové neváhali ani minutu a rozběhli se po zledovatělém chodníku.

Waldfruch zavrávoral, podjela mu noha, dárek vypadl z rukou a posléze ho vlastním ramenem rozdrtil. Ležel tam na sněhu, sbíral střepy a barevný papír a nestačil se divit: ,,No to snad..."

,,Nestalo se ti nic?" bál se o svého kolegu Bludowský. Pád vypadal hrozivě. ,,Mně ne, ale ten dárek. Ještě když jsem ho kupoval, byla to salátová mísa a teď..." řekl Waldfrucht a ukázal střep. ,,Stará lahev od piva."

O pár minut později zažila Dárková balicí služba nepříjemné překvapení v podobě dvou naštvaných policistů.

V kanceláři bezpečnostní služby si pak vykutálený podvodník vzpomínal na to, jak do krásných balíčků dával bezcenné cetky a ještě si za to nechával platit. Dárky se pak chystal prodat na tržnici. Bludowský vstal znechuceně od stolu. ,,Proto nemám rád Vánoce."

{\em Rolničky, rolničky, kdopak Vám dal hlas?}

Bludowský seděl v tramvaji. Tramvaj se řítila z kopce do Podhradí.

Přestože tam byla také vánoční výzdoba, čtvrť opravdu vypadala jako doupě zločinu, za které byla pokládána. Bludowský přemýšlel proč.

Čím to je, že zrovna tady už přes sto let zločin jen kvete? Je to těmi hospodami, továrnami, či domy? Je to ve vzduchu, vodě, půdě či snad v omítce, nebo je to prostě v povaze lidí? Nebo se stejně krade, loupí a vraždí i v jiných čtvrtích? Statistiku sebou v tramvaji neměl.

Díval se z okna, zatímco tramvaj míjela hospody, herny, zastavárny a pojištění cizinců. Teda pokud to poslední v té azbuce přečetl správně.

Bludowský vystoupil. Odemkl dveře zchátralého činžáku a vystoupal po schodech. Odemkl byt, zul si boty a vstoupil. Nikdo ho nevítal. Jen stará lednička tiše vrčela.

{\em Nésém vám nóviny, póslouchéjte...}

Nazítří seděl Bludowský na policejní stanici a prohlížel si oběžník s fotkou mrtvé. Nebylo to zrovna to, co by chtěl na Štědrý den dělat. Ale co naplat, někdo to dělat musí.

,,Co je zač?" zeptal se. ,,Zatím nevíme. Včera večer skočila pod vlak kousek za Městem. Řeší to kolegové z venkova" vysvětlil strážník. ,,A on?" Bludowský otočil hlavu na zadrženého. ,,Načerno prodával rachejtle na vánočních trzích, a když jsme ho tu vyslýchali, uprostřed řeči vstal, díval se na nástěnku, a ptal se kdo je ta žena. Je to první svědek, který ji poznal."

Svědek nebyl zrovna nadšený z toho, že s nim chce na Štědrý den mluvit kriminálka. ,,O vás mi nejde" uklidnil ho Bludowský ,,jde mi o tu ženu. Vy ji znáte?" - ,,Ne." opáčil svědek ,,jenom prodávala ve vedlejším stánku vánoční stromečky. Trochu mě zaujala, chtěl jsem ji pozvat na kafe, ale neodvážil jsem se." Bludowský se zamračil. Neměl rád sebevraždy. ,,Znal jste ji jménem?" - svědek odpověděl zavrtěním hlavou a detektiv vzdychl. Svědek vykulil oči: ,,Teda doufám, že si nemyslíte, že jsem jí něco udělal?" Bludowský si to nemyslel. Nemyslel si vůbec nic.

Výslech se ukázal jako zbytečný, svědek nic jiného nevěděl.

{\em My tři králové, jdeme k vám, štěstí zdraví, vinšujem vám....}

Magistrátní úřednice nebyla ráda, že ji na Štědrý den od pečení posledního cukroví vytáhnla kriminálka. Naštvaně hledala v počítači účtenky za nájem stánku na vánočních trzích. ,,Tady to je" mračila se na Bludowského ,,Žaneta Kmínová, Stoupavá ulice 32, Klikohrad. Nájem za stánek číslo 5. Předmět prodeje: vánoční stromečky." Bludowský si adresu a jméno pečlivě zapsal a slušně poděkoval úřednici za ochotu. Nebyla z toho nadšená. Bludowský věděl, že ani kolegové z Klikohradu nebudou nadšeni.

A měl pravdu. Kolegové v Klikohradu nadšeni nebyli. Nakonec se zjistilo, že dopis na rozloučenou ležel v poštovní schránce u Klikohradského nádraží. Tamní pošta měla o svátcích zavřeno, tak ji nikdo nevybral. V dopise si tato žena, jak se zjistilo později vdova po vojákovi, stěžovala své nejlepší přítelkyni na samotu a deprese. Na světě nebyl nikdo, kdo by ji podržel. Jedniný, kdo by to byl mohl zvrátit, byl ten černý prodejce rachejtlí, ale neměl odvahu.

Bludowský tohle všechno zatím nevěděl, ale stejně mu z toho bylo špatně.

{\em Narodil se Kristus Pán...}

Odpoledne měl Bludowský sraz na nádraží. Jeho bratr mu předával dárky: ,,To je od mámy, to je ode mě a Irenky. Je škoda, že si nás nemohl přijet navšívit." - ,,No jo, mám službu" povzdechl si Bludowský a pokračoval: ,,Pozdravuj mámu a Irenu. Předej dárky a nekoukej se ve vlaku na ten svůj!" - ,,Neboj, nejsme už přece malí." Oba se tomu zasmáli, jako se smáli svým vtipům, když byli malí. Pak nádražní rozhlas vyhlásil vlak a oba bratři se rozloučili. Bludowský potom zamíril do kanceláře, kde měl ještě nějaké papírování související s koncem roku.

{\em Štědrej večer nastal. Štědrej večer nastal, koledy přichystal...}

Štědrej večer nastal a Bludowský seděl doma. Společnost mu dělala vrnící lednička, lahev vína a napolo ztlumená televize. Koukal se z okna dalekohledem, Na plakátovací ploše na zdi vozovny vylepil kdosi plakát s nápisem Merry crysis and happy new fear.

To docela sedí. Jenom za dva poslední dny se setkal s jednou krádeží, jedním podvodem a sebevraždou.

Jestli tohle není krize, tak co teda?

Biskup v televizi popřál lidu Veselé Vánoce a začala pohádka. Bludowský vstal, nalil si víno, a neměl si s kým připít. Na chvilku se zarazil ale pak rozbalil dárky. Pletené ponožky, ty budou asi od mámy, a loď v láhvi, ta je od bratra. Žádné překvapení. Jako každý rok.

Bludowský vypnul televizi. Na pohádky neměl náladu.

Všichni ti zloději, podvodníci, vrazi, mafiáni a korupčníci sedí v kruhu rodinném, akorát on je sám doma. To má za svou celoroční pilnou službu - ani si o Vánocích nemůže vybrat dovolenou, aby navštívil příbuzné. Byl sám.

Pomalu přemýšlel, jestli takhle večer jezdí po trati nějaký milosrdný vlak, ale zazvonil zvonek. ,,Kdo to sakra takhle pozdě otravuje?"

Ve dveřích stál Waldfrucht s přítelkyní: ,,Promiň, dárky jsme nestačili koupit." - ,,Ale mysleli jsme, že ti naše návštěva udělá radost" přitakala Waldfruchtova přítelkyně Bety. ,,Pojďte dál"

Seděli, pili víno, povídali si. Svět přestal být černý. Pak zazvonil telefon. Nadšený hlas ve stroji děkoval: ,,Díky, brácho, to je ten nejlepší dárek cos mi kdy dal!" Bludowský se usmál, jeho bratr dodržel slib a nechal si dárek na doma.

Pak si dále povídali. O Vánocích, rodině, o tom, co kdo dostal. Bety se rozplývala nad krásnou ručně broušenou salátovou mísou.

,,A víš že jsem docela rád, že v té dárkové balící službě byl ten podvodník" podotkl Waldfrucht ,,jinak by teď ta mísa byla na padrť!"

,,Všechno zlé je k něčemu dobré" dodala Bety. Bludowský se usmál a smířil se světem. Nechal všechny zloděje a podvodníky, lichváře a politiky ať si v klidu užijí Vánoc. Však on jim to příští rok spočítá!!!

\page

\chapter{Tiráž}

První svazek {\rm \sc edice Pantograf}

Napsal a vydal {\rm \sc Mikoláš Štrajt}

Tato kniha je šířena pod licencí {\rm \sc CC-BY}. To znamená že smíte knihu šířit a adaptovat, ale nesmíte zapomenout uvést autora.

\emptylines[5]

Příště vyjde v edici Pantograf:

\emptylines[1]

{\framed[width=15cm,autowidth=force,align=no]

{\tfb Sokolí hnízdo}

\it{Bludowský si poručil ,,jedno" a zeptal se: ,,Co se to tam venku děje?" - ,,Ále, našli poklad" odpověděl Mozart a načepoval pivo a dodal: ,,už zase." - ,,Jaký poklad?" - ,,Ále, za války ho sem zakopali Němci, taky jsem ho zamlada hledával..."}

}

\stoptext