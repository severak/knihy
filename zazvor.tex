Jednu věc si ale uvědom: Ďasa si nevymejšlí pánbu. Čert se dycky rodí z lidkého hříchu, i když se ho lidi snažej zapudit. Jeden knihou, jinej ohněm, další eště něčím jiným... No a když ho konečně zažehnaj, a že se s tím nějak nějak natrápěj, tak skrze to svý trápení zploděj dalšího čerta - a zase to všecko běží vod začátku pěkně dokola....

-- S. Jaroslavcev: Ďábel mezi lidmi


Zázvor

aneb Malá (před|po)volební hudba

Učitel je pro studentky alfa samec.

Až se budete někdy rozčilovat nad podobným článkem, buďte v pozoru. Je to tím že ho napsal učitel. Stejně tak až vás budou plakáty po celém městě přesvědčovat, že  všichni volí Taškára, vězte, že byl Taškár zamlada předtím než se dal na politiku malíř, a všechny ty kreativce tak proto učili jeho spolužáci. A pak vás nemusí překvapit, že Kozák vyhrává průzkumy veřejného mínění. Studoval sociologii a absolventi sociologie právě ti ty průzkumy dělají.

Umělci dostávají místa v reklamě a sociologové falšují průzkumy trhu. Kam ten svět spěje?

Stejně tak není důvod proč důvěřovat městské lince vlaku. Vede ve směru, kam by normální člověk vlak nikdy neposlal. Proto většinou slouží train abusingu (nehodlám tu teď vysvětlovat co to je, sám to totiž často provozuju) a převozu podivných existencí z Jižního nádraží na Východní a zpět.

Markéta seděla a učila se na zkoušku z biologie o nějakých absurdních breberkách žijících ve slaném nálevu v továrně na kyselé okurky. Na Jižním nádraží přistoupili dva starší pánové, asi měli trochu upito. Ale to nebylo nezvyklé, tímhle posledním vlakem nikdo střízlivý nejezdí.

,,Večer si doma dám" začal jeden ,,Jak se tomu říká? Babička to dávala mě malýmu - takový kořínek je to...." Druhý vzpomínal: ,,Jo nedávno jsem si z toho dělal čaj..." - ,,Jak vono to jen...."

,,Zázvor" řekla do ticha Markéta. Kyselých okurek měla už po krk.

,,To je ale chytré děvče." pochválil ji jeden ,,zázvor je to" - ,,Ano. Ona je budoucnost tohoto města, my už jsme udělali dost." přizvukoval druhý.

Dvořit se uměli asi jako bojové vozidlo pěchoty náramkovým hodinkám a nejhorší na tom bylo to, že už ji dneska s tím že jí to sluší zastavoval pro změnu bezdomovec (jménem Leopold, který teď seděl na Jižním nádraží a kouřil cigarety značky Drina které vyžebral od jednoho Jugoslávce).

,,Když mi bylo dvacet, jezdila touhle tratí jedna moc pěkná průvodčí" mlel si svojí písničku první.

Markéta vzhlédla od nálevníků ve svých skriptech. Odkud jen ty dva dědky znala?

Vlak projel okolo domu kde byl plakát jednoho z nich v nadlidské velikosti,

No jistě! To jsou kandidáti na starostu! Už 14 dní tu bojujou o hlasy lidí jak dva rozzuření nosorožčí samci a teď je vidím vedle sebe družně hovořící ve vlaku. No potěš koště!