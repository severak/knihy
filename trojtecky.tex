%\setupoutput[pdftex]
\enableregime[utf]
\definetypeface[iwona][ss][sans][kurier][default]
\usetypescript
   [modern][ec]
\setupbodyfont
   [ss,10pt]
\mainlanguage
   [cz]
\setuppapersize
   [A4][A4]
\setupcolors
   [state=start]
\setuppagenumbering[location={footer,middle}]

\setuphead
   [chapter]
   [header=chapter,page=no,command=\green\iwona,number=no]

\def\citat #1#2{
{\it
#1
\blank[1*halfline]
\rightaligned{#2}
}}

\def\trojtecka {
{\bf \ruda \tt...}
}

\def\trojtecky{
\midaligned{\tt{\bf \green ***} }
}

%%% Odstavce %%%
\setupindenting[yes, small,first]
\setupheads[indentnext=yes]

%%% Záložky pdf %%%
\placebookmarks[chapter,section,subsection][force=yes]
\setupinteractionscreen[option=bookmark]
\setupinteraction[state=start,
  title={Trojtečky},
  author={Mikolas Strajt}]
\setuplist[chapter][interaction=text,color=indigo]

%%% COLORS %%%
\definecolor[indigo][r=.0, g=.0, b=.5]
\definecolor[green][r=0, g=0.5, b=0]
\definecolor[ruda][r=0.5, g=0, b=0]



\starttext
\startTEXpage
    \externalfigure[trojtecky-titul.pdf][page=1]
\stopTEXpage

\startstandardmakeup
{\sc \rm Mikoláš Štrajt}
\title{Trojtečky}
\placecontent
\stopstandardmakeup
\page

\chapter{Úvodem}

Psaní detektivek nepatří mezi nejlehčí činnosti kterými ztrácím volný čas.

Kromě věcí na které jsem zvyklý z povídek (pravopis, jednota nálady, nezapomenout něco důležitého) je třeba dát pozor i na uvažování detektivů, pohnutky podezdřelých, napínání čtenářů a celkovou uvěřitelnost. Jo a taky neporušovat pravidla pátera Knoxe.

Z toho plyne že dokončit detektivku není lehký úkol. Ale i ty rozepsané v sobě mají věty, které se mi povedly.

'spoň doufám...

Tímto spiskem bych vás tedy nechal nahlédnout do své, prachmaticky řečeno, vysoké pece.

Některé příběhy časem dopíšu, jiné zůstanou s trojtečkou na konci na vždy.

To už záleží jen na mě.

A na Vás\trojtecka

\rightaligned{Severák}

\blank[1*line]
\trojtecky
\blank[1*line]


\citat{V tomto šerosvitu, v němž už nikdo neví, kdo je kdo, se lidé cítí cize, a to nejen ve světě, ale také mezi sebou navzájem. A v náladě cizoty a osamělosti dostávají mezi lidmi svou příležitost postavy cizinců, světci a zločinci.}{Hannah Arendtová: Vita activa aneb o činném životě}

\page

\chapter{KOLDOM}

Stěny tunelu se otřásaly. Ze stropu se sypal písek. Ozývalo se dunění. Za chvíli prorazí. Prorazí nebo se sesype celý tunel. Ozvala se další rána a ze stropu se sesypala poslední hrst písku.

Prorazila!

Ruka prorazila a šátrala tunelem. Podal jsem jí tou dírou ruku a potřásli jsme si na počest dokončení tunelu.

Tak na pískovišti přibyl další tunel a začal příběh,který vám chci vyprávět.

Bylo mi tehdy deset let a jako každý rok jsem o prázdninách jel k babičce a dědečkovi. Něco však bylo jinak. U mého kamaráda Michala bydlela to léto jeho sestřenice Lucie. Ostatní holky z baráku s ní nechtěly mít nic společného, tak Lucie chodila pořád s námi. Nakonec jsme museli uznat, že neni tak otravná a alespoň umí stavět tunely.

Zrovna jsme budovali nájezdovou rampu, když se z okna ozvalo: ,,Milane! Večeře!"

Paní Filipová! ,,Máma!" křikl Milan.

Zakonzervovali jsme stavby, uklidili bagry do igelitky a vyrazili domů. Minuli jsme paní Mackovou ve vrátnici: ,,No tak, kluci, nedupte tu ten písek!" A běželi jsme dál po schodech.

Ano schody. Schody byly snad to nejzajímavější z celého domu. A to je to hrozně zajímavý dům, dnes je kulturní památka. Každopádně schody byly zajímavé tím, že jako spojnice s mezipatrem byl krátký protisměrný úsek. Běželi-li jsme nahoru, museli jsme seběhnout deset schodů dolu.

Seběhl jsem těch deset schodů. ,,Tady běhaj po schodech nějaký blázni" smáli jsme se. Vtom na nás vykoukla ze dveří nějaká paní: ,,Děcka, to nemůžete jezdit výtahem?" Běželi jsme dál.

Přišel jsem udýchaný domů. Bytem zněl cvakot dědova psacího stroje. ,,Ahoj" řekl jsem. Babička nakoukla se skleničkou a útěrkou v ruce: ,,Nemůžete takhle běhat po schodech.  Před chvílí si sem šla stěžovat paní Průchová, že jste jí rozsypali hlínu z květináčů." - ,,Ale my jsme to nebyli" - ,,Tak se už naučte jezdit výtahem ať si nestěžuje"

Následující ráno na bojové poradě na prolézačce jsme se schodli: ,,Stěžovala si na nás stará Průchová, ale já s tim nemám nic společnýho..." - ,,A co s tim uděláme?" zeptala se Lucie. ,,A co s tim jako chceš dělat Sherlocku?" - ,,To je prosté - milý Watsone - vyšetřit to!"

Domem se rozléhala příšerná houslová kakofonie. ,,Mirku" řekla babička dědovi ,,nezdá se ti, že to ten sousedovic Milan dneska nějak přehání s tim cvičením houslí? Normálně ho Maruška musí nutit."

Ale Milan to nepřeháněl. Přeháněla to Lucka. S houslemi pod bradou a s dýmkouv puse se vžívala do role Sherlocka.

,,Máš už nějaký nápad?" - Lucka přestala hrát a zamyšleně prohodila: ,,Základem vyšetřování je dobrý převlek, třeba žebrák.\footnote{Tady je asi dobré přpomenout že se příběh odehrává za dob komunismu, kdy žebráci nepatřili k základní výbavě města.} Aby nikdo nepoznal že vyšetřujem." Milan se plácl do čela: ,,V převleku za žebráky budem fakt nenápadný..." Lucka se zamyslela. ,,Mám to! Převlíknem se za děti!" Podívali jsme se s Milanem na sebe. Ani trochu jsme nerozuměli.

,,Je to prosté milý Watsone! Nikdo nebude děti podezdřívat z toho, že něco vyšetřují. Budem třeba předstírat, že si hrajem na schovku, a ve skutečnosti budem pozorovat a poslouchat" vysvětlila. ,,Sherlocku, ty jsi vážně geniální" Lucie se ďábelsky zašklebila. ,,Díky"

Paní Macková seděla ve vrátnici\trojtecka

\page

\chapter{Dvojí život Áni Kellerové}

Áňa Kellerová\footnote{Než se zeptáte: ano, mám na to povolení od Áni, dokonce mě povzbuzovala ať to dopíšu...} ztratila pas. To se tak občas stává, problém byl v tom že ho ztratila na cestách. A to by se lidem stávat nemělo.

Byl krásný letní večer. Áňa ležela v trávě v ruce láhev vína, v hlavě tisíc zvonečků. Takový ten okamžik kdy je člověku smutno, a veselo zárověň. Vzduchem se nesl smích smíchaný s hudbou.

Tento příběh však není o Áně. Je jen o jejím jméně, fotografii, datu narození a bydlišti. Hlavním hrdinou tohoto příběhu je totiž její cestovní pas\trojtecka

\page

\chapter{Čas lovu}

Nastal čas lovu. Polem kráčela smrt. Měla podobu můžů v zeleném s flintami a psy. Ozýval se hukot trubek, štěkot psů a rány výstřelů. Zvířata prchala hluboko do lesa.

Najednou vše ustalo a nad zoranými poli se ozýval pouze tichý pláč. Jeden z lovců seděl zhroucený na zemi, u nohou ještě teplá flinta, a usedavě plakal: ,,Ne, já vážně nechtěl. Néé! Néé! Néé!"

Za dva dny na to místo mířil Waldfrucht. Měl k tomu docela dobrý důvod - místní policie na tom místě našla mrtvolu, kterou posléze identifikovali jako Petra Baumana, muže po kterém pátral už půl roku.

Waldfrucht jel po hlavní silnici. Podle pokynů zajel ve vesnici Čížkovice do prostoru autobusové zastávky a stáhnul okýnko. Z davu čekajícího na autobus 

\chapter{Na počest hrdinům}

...

\chapter{Sokolí hnízdo}

...

\chapter{Problém tří těles}

...

\chapter{Pískáček}

...

\stoptext